%pdflatex -halt-on-error -aux-directory=tmp -output-directory=tmp rapport.tex%

\documentclass{article}
\usepackage{amsmath}
\usepackage[utf8]{inputenc}
\usepackage[T1]{fontenc}
\usepackage{graphicx}
\usepackage{hyperref}
\usepackage[francais]{babel}
\usepackage{listings}
\usepackage{xcolor}

\definecolor{codegreen}{rgb}{0,0.6,0}
\definecolor{codegray}{rgb}{0.5,0.5,0.5}
\definecolor{codepurple}{rgb}{0.58,0,0.82}
\definecolor{backcolour}{rgb}{0.95,0.95,0.92}

\lstdefinestyle{mystyle}{
    language=python,
    backgroundcolor=\color{backcolour},   
    commentstyle=\color{codegreen},
    keywordstyle=\color{magenta},
    numberstyle=\tiny\color{codegray},
    stringstyle=\color{codepurple},
    basicstyle=\ttfamily\footnotesize,
    breakatwhitespace=false,         
    breaklines=true,                 
    captionpos=b,                    
    keepspaces=true,                 
    numbers=left,                    
    numbersep=5pt,                  
    showspaces=false,                
    showstringspaces=false,
    showtabs=false,                  
    tabsize=2
}

\lstset{style=mystyle}

\title{Rapport sur le TP de probabilité et statistiques}
\author{Wassim Saidane, Aurélien Authier}
\date{}

\begin{document}
    \pagenumbering{gobble}
    \maketitle
    \tableofcontents
    \newpage
    \pagenumbering{arabic}
    \section*{Introduction}
    Dans ce TP nous allon modéliser différents problèmes de probabilité et statistique en python
    \section{Première partie : Régression linéaire}
    Cette partie correspond au fichier AuthierSaidanePart1.py
    \subsection{Régression Linéaire simple}
    Soit le modèle linéire suivant : 
    \begin{equation*}
        Y_1=\beta_0+\beta_1x_i
    \end{equation*}
    Avant de calculer le coeficient de régression nous avons créer une fonction moyenne.
    \begin{lstlisting}
        def moyenne(tab):
            return sum(tab)/len(tab)
    \end{lstlisting}
    Ainsi nous pouvons faire la moyenne de différentes valeurs. 
    Une fois la fonction moyenne codée on peut faire la fonction les coeficients de régressions.
    Voici la formule du coeficient de régression : 
    \begin{equation*}
        a=\frac{\sum^{n}_{i=1} (x_i-\bar{x})(y_1-\bar{y})}{\sum^{n}_{i=1} (x_i-\bar{x})^2}
    \end{equation*}
    \begin{equation*}
        b=\bar{y}-a\bar{x}
    \end{equation*}
    En python celà se traduit par : 
    \begin{lstlisting}
        def coef_regression(tab1, tab2):
            beta_1 = 0
            temp1 = 0
            beta_0 = 0
            temp2 = 0
            for i in range(0, len(tab1)):
                temp1 += (tab1[i]-moyenne(tab1))*(tab2[i]-moyenne(tab2))
                temp2 += (tab1[i]-moyenne(tab1))**2
            beta_1 = temp1/temp2
            beta_0 = moyenne(tab2)-beta_1*moyenne(tab1)
            return (beta_0, beta_1)
    \end{lstlisting}
    Les deux tableaux en paramètres de notre fonction sont les échantillons du sujet.
    beta\_0 correspond à $b$ et beta\_1 correspond à $a$.
    On obtient : \\
    b=0.5733333333333334 \\
    a=16.799999999999997 \\
    \newpage
    Pour la représentation graphique nous avons utilisé l'outil \href{https://matplotlib.org/api/pyplot_api.html}{pyplot} de la bibliothèque \href{https://matplotlib.org}{matplotlib}.
    \begin{lstlisting}
        coefs = coef_regression(x_i, y_i)
        x = np.linspace(40, 100)
        plt.plot(x, coefs[0]+coefs[1]*x)
        plt.scatter(x_i, y_i, label="Nuage de points")
        p = np.polyfit(x_i, y_i, 1)
        plt.plot(x, p[1]+p[0]*x, label="Droite de regression")
        plt.legend()
        plt.title("Nuage de points et droite de regression")
        plt.show()
    \end{lstlisting} 
    On obtient le graphique suivant : \\
    \includegraphics{regre.png}
    \newpage
    Soit le modèle vectoriel suivant : 
    \begin{equation*}
        y=A\beta
    \end{equation*}
    Pour pouvoir faire des opérations sur les matrices nous avons utilisé la bibliothèque \href{https://numpy.org}{numpy}.
    \\ Nous avons coder la formule $(A^TA)^{-1}A^Ty)$. \\
    Tout d'abord nous avons implémenter deux petites fonctions permettant de determiné $A$, $A^T$ et $Y$.
    \begin{lstlisting}
        def matA(x):
            vect = np.ones(len(x))
            Atranspose = np.vstack((vect, x))
            A = np.transpose(Atranspose)
            return (A,Atranspose)
        def matY(y):
            return np.matrix(y_i).transpose()
    \end{lstlisting}
    Ensuite en deux opérations nous avons calculer $A^TA$ et $(A^TA)^-1$.
    \begin{lstlisting}
        A_transposeA = np.dot(A_transpose, A)
        inv_ATA = np.linalg.inv(A_transposeA)
    \end{lstlisting}
    Pour finir nous avons implémenter la formule demandé : 
    \begin{lstlisting}
        def matBeta(inv,Atranspose,Y):
            B1=np.dot(inv,Atranspose)
            B2=np.dot(B1,Y)
            return np.asarray(B2).reshape(-1)
    \end{lstlisting}
    On obtien ainsi Beta : [16.8         0.57333333] \\
    Ansi que la graphique suivant : (voir page suivante)\\
    \includegraphics{regre2.png}
    \newpage
    \subsection{Régression linéaire et descente de gradient}
    \newpage
    \section{Deuxième partie: Etude et manipulation de lois de probailités}
    \subsection{Loi Binomiale}
    Nous avons modéliser en python une loi binomiale, pour celà nous avons utiliser la fonction python \href{https://koor.fr/Python/API/scientist/numpy.random/binomial.wp}{np.random.binomial}.
    \begin{lstlisting}
        def binomial (n,p):
            valeur = []
            binom = []
            for i in range(100):
                binom.append(np.random.binomial(n, p))

            for i in range(100):
                valeur.append(binom.count(i))
    
        return valeur
    \end{lstlisting}
    Enusuite on ajoute les 3 expériences.
    \begin{lstlisting}
        exp1 = binomial(30, 0.5)
        exp2 = binomial(30, 0.7)
        exp3 = binomial(50, 0.4)
    \end{lstlisting}
    On obtient ainsi le résultat suivant : (page suivante)\\
    \includegraphics{LoiBinom.png}
    \newpage
    \subsection{Loi Normale univariée}
    \subsection{Simulation de données à partir d'une loi}
    \subsubsection{Cas de la loi Normale}
    \subsection{Estimation de densité}
    \subsubsection{Cas de la loi Normale}
    \subsubsection{Cas de la loi exponentielle}
    \newpage
    \section{Troisième partie: Intervalle de confiance}
    \subsection{Probléme 1}
    On calcul la moyenne empirique : 
    \begin{lstlisting}
        moyenne = sum(poids)/len(poids)
    \end{lstlisting}
    Ensuite on affiche l'histogramme grace aux fonctions python et on obtien : \\
    \includegraphics{hist.png} 
    \newpage
    On calcul l'écart type : 
    \begin{lstlisting}
        def ecart_type_emp(data):
            moyenne = sum(data)/len(data)
            somme = 0
            for valeur in data:
                somme += (valeur-moyenne)**2
            return sqrt(somme/(len(data)-1))
    \end{lstlisting}
    L'intervalle à 95\% et 99\% on utilise les formules classisques. Par exemple, pour l'intervalle à 95\%,
    \begin{lstlisting}
        intervalC95 = [moyenne - interval95*et/sqrt(len(poids)),
        moyenne + interval95*et/sqrt(len(poids))]
    \end{lstlisting}
    \subsection{Probléme 2}
    Pour calculer l'intervalle de confiance : 
    \begin{lstlisting}
        n = 500
        proba = 95/n
        ecart_type = sqrt(proba*(1-proba)/n)
        quantile = 2.576

        intervalC99 = [proba-quantile*ecart_type/sqrt(n), proba+quantile*ecart_type/sqrt(n)]
        print(f"Compagnie : Interval a 99% = {intervalC99}")
    \end{lstlisting}
    La proportion d'individus satisfaits de la compagnie aérienne est comprise entre 18.8\% et 19.2\%. 
    \subsection{Probléme 3}
    Nous avons pris un effectif de 100.
    \begin{lstlisting}
        n = 100
        binomiale = np.random.binomial(n, 0.5, 500)
        ecart_type = sqrt(100*0.5*(1-0.5))
        moyenne = sum(binomiale)/len(binomiale)

        intervalC95 = [moyenne - 1.96*ecart_type/sqrt(n), moyenne + 1.96*ecart_type/sqrt(n)]
        print(f"Binomiale : interval a 95% = {intervalC95}")
    \end{lstlisting}
\end{document}
